%Beschreibung der Implementierung der SPH-Flüssigkeitssimulation

\section{Partikel-basierte Simulation}
Es gibt zwei grundlegende Ansätze für die Simulation von Flüssigkeiten: entweder auf Gittern oder auf Partikeln basierend. Die Repräsentation von Simulationen in einem Gitter ist genauer, jedoch ebenso rechenintensiver \cite{fluid.tutorial_paper}. Da dieses Projekt auf Mikrocontrollern umgesetzt und die Flüssigkeit auf einem Bildschirm mit niedriger Auflösung angezeigt werden soll fiel die Wahl auf die Partikel-basierte Simulation.

\subsection{Grundlagen}

\subsection{Umsetzung}

\section{Rechenbeschleunigung durch RaspberryPi 1B}