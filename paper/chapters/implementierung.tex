%Einleitung zur Implementierung
%Sprache, Stiel, Debugging, IDE, etc.
\section{Entwicklungsumgebung}
Zur Entwicklung der Softwarekomponente soll erst Keil\textsuperscript{\scriptsize\textregistered} Microcontroller Development Kit (MDK) auf einem Rechner installiert werden. Das Kit ist besonders gut geeignet für Arm\textsuperscript{\scriptsize\textregistered}-basierte Microcontroller und beinhaltet alle notwendigen Komponente für Embedded Applikationen. Die Edition MDK-Lite ist konstenlos jedoch mit der Beschränkung von 32KBytes Codegröße verfügbar. Zusätzlich muss Legacy Package ARMv7 (NXP LPC 2388) installiert werden. Anschließend soll die Integrierte Entwicklungsumgebung Keil $\mu$version 5 erfolgreich installiert sein.

Der Board kann durch einen USB-Anschluss vom Rechner eingeschaltet werden. Der ULINK-ME kann ebenfalls durch einen USB-Anschluss zum Laden des Programms und zum Debuggen verwendet werden.

%Beschreibung der Konfiguration von IDE
