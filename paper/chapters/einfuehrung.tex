Im Rahmen des Moduls `Fortgeschrittene ARM Programmierung' wird nutzen wir das Keil Evaluation Board MCB2388, welches einen NXP LPC2388 Microcontroller und verschiedene Schnittstellen wie LCD, USB, UART oder Ethernet besitzt. Der ARM7-Prozessor auf dem Board kann mit der mit bis zu 72 MHz getaktet werden. Die vielen verschiedenen Schnittstellen, die das Board mitbringt, ermöglichen eine große Breite an verschiedenen Projekten.

Insbesondere unter Berücksichtigung von seiner im Vergleich zu anderen uC ähnlicher Bauart hohen Taktrate sind wir zur Idee gekommen Flüssigkeiten zu simulieren und in 3D auf einem aus 6 LED-Panelen aufgebauten Würfel zu visualisieren. Zur Realisierung unseres Projekts wird eine sehr hohe Taktrate benötigt werden, um die Bewegung der Wasserteilchen auszurechnen und diese auf den LED Panels in Echtzeit darzustellen. Zusätzlich zu dem Entwicklungsboard sowie zu den Panelen wird ein Beschleunigungssensor benötigt, um die 3D Ausrichtung des Würfels zu bestimmen und die Flüssigkeit entsprechend zu bewegen.

