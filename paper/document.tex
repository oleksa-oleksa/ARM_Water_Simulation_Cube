
\documentclass[12pt, a4paper]{book}

%% Übersetzen als Entwurf
%\usepackage[entwurf]{bhtThesis}
%% Übersetzen für die Abgabe
\usepackage[abgabe]{bhtThesis}
\let\ifpdf\relax	%bhtThesis nutzt \ifpdf

\usepackage{blindtext}   %für Blindtext in Kapitel 2

\usepackage{listings}             % Include the listings-package
\renewcommand\lstlistlistingname{Quelltextverzeichnis}	%Übersetzung
\renewcommand{\lstlistingname}{Quelltext}	%Übersetzung

\newcommand*\ttvar[1]{\texttt{\expandafter\dottvar\detokenize{#1}\relax}}
\newcommand*\dottvar[1]{\ifx\relax#1\else
	\expandafter\ifx\string_#1\string_\allowbreak\else#1\fi
	\expandafter\dottvar\fi}

\usepackage{xcolor}
\usepackage{xparse}% to define star variant of macro
\makeatletter
\def\lst@MSkipToFirst{%
	\global\advance\lst@lineno\@ne
	\ifnum \lst@lineno=\lst@firstline
	\def\lst@next{\lst@LeaveMode \global\lst@newlines\z@
		\lst@OnceAtEOL \global\let\lst@OnceAtEOL\@empty
		\ifnum \c@lstnumber>0
		\vspace{2 mm}
		\fi
		\lst@InitLstNumber % Added to work with modified \lsthk@PreInit.
		\lsthk@InitVarsBOL
		\c@lstnumber=\numexpr-1+\lst@lineno % this enforces the displayed line numbers to always be the input line numbers
		\lst@BOLGobble}%
	\expandafter\lst@next
	\fi}
\makeatother

\lstset{ 
  literate={ö}{{\"o}}1
           {ä}{{\"a}}1
           {ü}{{\"u}}1
           {Ö}{{\"O}}1
           {Ä}{{\"A}}1
           {Ü}{{\"U}}1
           {ß}{{\ss}}1,
%  language=C,
  numbers=left,
  stepnumber=1,
  tabsize=2,
  breaklines=true,
  basicstyle=\ttfamily\footnotesize,
  columns=fullflexible,
  frame=single,
  alsoother={_-},
%  breakatwhitespace=false,
  alsoletter={()[]=},
  postbreak=\mbox{\textcolor{red}{$\hookrightarrow$}},
  keepspaces=true
}

\usepackage{multirow}
\usepackage{array}
\newcolumntype{L}[1]{>{\raggedright\let\newline\\\arraybackslash\hspace{0pt}}m{#1}}
\newcolumntype{C}[1]{>{\centering\let\newline\\\arraybackslash\hspace{0pt}}m{#1}}
\newcolumntype{R}[1]{>{\raggedleft\let\newline\\\arraybackslash\hspace{0pt}}m{#1}}

\usepackage[breaklinks=true]{hyperref}

\usepackage{textcomp}

\usepackage{wrapfig}	%wrapfigure

\renewcommand{\textdownarrow}{$\downarrow$}
\renewcommand{\textleftarrow}{$\leftarrow$}

\usepackage{pdfpages}	%includepdf
\usepackage{grffile}	%spaces in filenames
\usepackage{chngcntr}	%continous footnote numbering (no reset from chapter to chapter)
\counterwithout{footnote}{chapter}	%setting for package chngcntr
\usepackage[n, operators, logic]{cryptocode}	%for crypto, pseudocode and protocol flow graphs
\usepackage{placeins}	%for \FloatBarrier
%\usepackage{float}		%for H option for floats
\usepackage{caption}	%captionof in minipage und source (siehe unten)

\newcommand{\source}[1]{\vspace{-10pt}\caption*{{#1}}}

\usepackage{xargs}                      % Use more than one optional parameter in a new commands
\definecolor{OliveGreen}{cmyk}{0.64,0,0.95,0.40}
\definecolor{Plum(traditional)}{rgb}{0.56, 0.27, 0.52}
\usepackage[colorinlistoftodos,prependcaption,textsize=tiny]{todonotes}
\newcommandx{\unsure}[2][1=]{\todo[linecolor=red,backgroundcolor=red!25,bordercolor=red,#1]{#2}}
\newcommandx{\change}[2][1=]{\todo[linecolor=blue,backgroundcolor=blue!25,bordercolor=blue,#1]{#2}}
\newcommandx{\info}[2][1=]{\todo[linecolor=OliveGreen,backgroundcolor=OliveGreen!25,bordercolor=OliveGreen,#1]{#2}}
\newcommandx{\improvement}[2][1=]{\todo[linecolor=Plum(traditional),backgroundcolor=Plum(traditional)!25,bordercolor=Plum(traditional),#1]{#2}}
\newcommandx{\thiswillnotshow}[2][1=]{\todo[disable,#1]{#2}}

\usepackage{microtype}	%better typesetting
\usepackage[nopostdot, acronyms, nonumberlist, style=altlist, toc]{glossaries} %glossary, option: no location within the glossary %section=chapter, 
%\usepackage{glossaries-extra} %print all unused glossary entries
\makeglossaries
\loadglsentries{glossary}
\glsaddall


\hyphenation{Schlüs-sel-aus-tausch hash-en-den} %Worttennungen

\usepackage[]{appendix}

\usepackage{dirtree} %directory tree

\usepackage[bottom]{footmisc} %figures not below footnotes

%Bibliographie:
\usepackage[numbers, round]{natbib}
\bibliographystyle{alphadin}

%Flowcharts
\usepackage{tikz}
\usetikzlibrary{shapes.geometric, arrows}

\tikzstyle{startstop} = [rectangle, rounded corners, minimum width=3cm, minimum height=1cm,text centered, draw=black, fill=red!30]
\tikzstyle{io} = [trapezium, trapezium left angle=70, trapezium right angle=110, minimum width=3cm, minimum height=1cm, text centered, text width=3cm, draw=black, fill=blue!30]
\tikzstyle{process} = [rectangle, minimum width=3cm, minimum height=1cm, text centered, text width=3cm, draw=black, fill=orange!30]
\tikzstyle{decision} = [diamond, minimum width=3cm, minimum height=1cm, text centered, text width=3cm, draw=black, fill=green!30]
\tikzstyle{arrow} = [thick,->,>=stealth]
%Flowcharts end


%%
%% Es folgen einige Zusätze, die in Kapitel 1 beschriben sind. 
%% Alles was nicht notwendig ist, kann auskommentiert werden
%%
\usepackage{trsym}
\usepackage{bytefield}

%%
%% Use git metadata
%%
\usepackage[missing={gitHeadInfo.gin not found!}]{gitinfo2}

%%
%% Pfad zu den Bildern
%%
\graphicspath{
  {pictures/}
}

%%
%% Persönliche macros 
%%


%% Message
\typeout{---------------------------------------------------------------}
\typeout{----> document.tex ---- Zentrales Dokument---------------------}
\typeout{---------------------------------------------------------------}

%TODO: add gitinfo
\version{}
\datum{\today}
%%
%% Titel, Autor und Betreuer
%%
\fachbereich{VI -- Informatik und Medien --} 
\studiengang{Technische Informatik - Embedded Systems}
\autor{Kayoko Abe, Oleksandra Baga, Heiko Radde}
\edvnr{826058, 849852, 887027}
\titel{Dynamische Flüssigkeitssimulation auf einem RGB-Würfel}

%%\renewcommand{\baselinestretch}{1.05} 
\begin{document}
\pagestyle{fancy}
%\pagenumbering{empty}

%%
%% Beuth Hochschule für Technik --  Abschlussarbeit
%%
%% Titelseiten und Erklärungen 
%%
%%%%%%%%%%%%%%%%%%%%%%%%%%%%%%%%%%%%%%%%%%%%%%%%%%%%%%%%%%%%%%%%%%%%%

\pagenumbering{roman}
\maketitle
\clearpage
%\thispagestyle{empty}
% Rueckseite (leer)
%% 
~
%\newpage
%
%\section*{Kurzfassung}
%Die Kurzfassung gibt ein kurzes und prägnantes Bild der gesamten Arbeit. Sie soll den Leser neugierig machen und klarmachen, was zu erwarten ist. Erreichte Ergebnisse werden kurz umrissen.



%\clearpage
%
%\section*{Erklärung}
%Ich  versichere, dass ich diese Abschlussarbeit ohne fremde Hilfe selbstständig verfasst und nur die angegebenen Quellen und Hilfsmittel benutzt habe. Wörtlich oder dem Sinn nach aus anderen Werken entnommene Stellen sind unter Angabe der Quellen kenntlich gemacht.
%\vspace{10ex}\\
%
%\hrule
%{\small{14.08.2018}}\hfill{\small{Unterschrift}}

\microtypesetup{protrusion=false}
\tableofcontents
\listoffigures
\lstlistoflistings
\cleardoublepage
\pagenumbering{arabic}
\microtypesetup{protrusion=true}
%\listoftodos[Notes]


%%%%%%%%%%%%%%%%%%%%%%%%%%%%%%%%%%%%%%%%%%%%%%%%%%%%%%%%%%%%%%%
%% Die Kapitel der Arbeit

%\todo{Todo}
%\unsure{unsure}
%\change{change}
%\info{info}
%\improvement{improvement}

\pagenumbering{arabic}

\clearpage
\chapter{Einführung}
\label{chap:einfuehrung}
%verbale Formulierung der Aufgabenstellung, erste Abschätzung der Aufwände
Im Rahmen des Moduls `Fortgeschrittene ARM Programmierung' wird nutzen wir das Keil Evaluation Board MCB2388, welches einen NXP LPC2388 Microcontroller und verschiedene Schnittstellen wie LCD, USB, UART oder Ethernet besitzt. Der ARM7-Prozessor auf dem Board kann mit der mit bis zu 72 MHz getaktet werden. Die vielen verschiedenen Schnittstellen, die das Board mitbringt, ermöglichen eine große Breite an verschiedenen Projekten.

Insbesondere unter Berücksichtigung von seiner im Vergleich zu anderen uC ähnlicher Bauart hohen Taktrate sind wir zur Idee gekommen Flüssigkeiten zu simulieren und in 3D auf einem aus 6 LED-Panelen aufgebauten Würfel zu visualisieren. Zur Realisierung unseres Projekts wird eine sehr hohe Taktrate benötigt werden, um die Bewegung der Wasserteilchen auszurechnen und diese auf den LED Panels in Echtzeit darzustellen. Zusätzlich zu dem Entwicklungsboard sowie zu den Panelen wird ein Beschleunigungssensor benötigt, um die 3D Ausrichtung des Würfels zu bestimmen und die Flüssigkeit entsprechend zu bewegen.



\chapter{Implementierung}
%Einleitung zur Implementierung
%Sprache, Stiel, Debugging, IDE, etc.
\section{Flüssigkeitssimulation}
%Beschreibung der Implementierung der SPH-Flüssigkeitssimulation

\section{Partikel-basierte Simulation}
Es gibt zwei grundlegende Ansätze für die Simulation von Flüssigkeiten: entweder auf Gittern oder auf Partikeln basierend. Die Repräsentation von Simulationen in einem Gitter ist genauer, jedoch ebenso rechenintensiver \cite{fluid.tutorial_paper}. Da dieses Projekt auf Mikrocontrollern umgesetzt und die Flüssigkeit auf einem Bildschirm mit niedriger Auflösung angezeigt werden soll fiel die Wahl auf die Partikel-basierte Simulation.

\subsection{Grundlagen}

Die Simulation von Flüssigkeiten beruht auf den Navier Stokes Gleichungen \cite{fluid.tutorial_paper}\footnote{$\overrightarrow{u}$ ist die Geschwindigkeit der Flüssigkeit, $\rho$ die Dichte. Der Druck innerhalb der Flüssigkeit wird durch $p$ dargestellt und $v$ entspricht der kinematische Viskosität}: 

\begin{align}
	\frac{\partial \overrightarrow{u}}{\partial t} + \overrightarrow{u} + \frac{1}{\rho} \nabla p &= \overrightarrow{F} + v \nabla \cdot \nabla \overrightarrow{u} \\
	\nabla \cdot \overrightarrow{u} &= 0
\end{align}

\cite{fluid.website} und \cite{fluid.website2} führt aus, wie mit Hilfe der Kernel \texttt{poly6}, \texttt{spiky} und \texttt{viscosity} die folgenden Formeln für die Dichte, den Druck und die Viskosität erstellt werden können:

\begin{align}
	\rho_i &= \sum^{j} m_j W(|r_i - r_j|, h) \\
	F_i^{pressure} &= - \sum^{j} m_i m_j (\frac{p_i}{\rho_j^2} + \frac{p_j}{\rho_j^2}) \nabla W(|r_i - r_j|, h) \\
	F_i^{viscosity} &= \eta \sum^{j} m_j \frac{|u_j - u_i|}{\rho_j} \nabla^2 W(|r_i - r_j|, h)
\end{align}

Für die einzelnen Kernel gilt:
\begin{align}
	W_{poly6}(r, h) &= \frac{315}{56 \pi h^9} 	\begin{cases}
													(h^2 - r^2)^3,   & 0 \leq r \leq h\\
													0, & \text{andernfalls}\\
												\end{cases} \\
	W_{spiky}(r, h) &= \frac{15}{\pi h^6}		\begin{cases}
													(h - r),   & 0 \leq r \leq h\\
													0, & \text{andernfalls}\\
												\end{cases} \\
	W_{viscosity}(r, h) &= \frac{15}{2 \pi h^3}		\begin{cases}
													-\frac{r^3}{2h^3} + \frac{r^2}{h^2} + \frac{h}{2r} - 1,   & 0 \leq r \leq h\\
													0, & \text{andernfalls}\\
												\end{cases}
\end{align}

Ein Problem in der Simulation ist, dass alle Partikel auf alle anderen einwirken. Dies stellt eine $O(n^2)$ Beziehung da, die sich sehr negativ auf den Rechenaufwand auswirkt. Ein Weg die Berechnung zu Beschleunigen beruht auf dem Fakt, dass der Einfluss von entfernteren Partikeln auf das gerade betrachtete stetig abnimmt. Wenn man nun die aktuelle Position der Partikel auf einem Schachbrett aufträgt, so reicht es nur die Partikel, die sich in den umliegenden Feldern befinden, in die Berechnung der Kräfte die auf einen gegebenen Partikel wirken, zu betrachten.

\subsection{Umsetzung}

Die Simulation der Partikel wurde in \texttt{C} für den 2D-Raum umgesetzt. In einem 2D-Array sind in jedem Eintrag doppelt verkettete Listen aus den Partikeln verlinkt, die sich aktuell innerhalb dem Feld befinden (siehe \ref{code.sph.structs}). Zur Berechnung der nächsten Position eines jeden Partikels werden Dichte, Druck und Viskosität des Partikels unter Berücksichtigung der Umliegenden berechnet. Addiert man nun die soeben berechneten Kräfte mit der extern Einwirkenden (Erdanziehungskraft) und integriert dies über die verstrichene Zeit erhält man den Beschleunigungsvektor des Teilchens. Die Multiplikation der Beschleunigung mit der verstrichenen Zeit ergibt nun die Positionsveränderung des Partikels.

\begin{lstlisting}[language={c}, caption={Datenstrukturen der SPH}, label={code.sph.structs}]
typedef struct
{
	double position[2];
	double velocity[2];
	double mass;
	double density;
	double pressure;
	double force[2];
} particle_t;


typedef struct particle_list_element_s
{
	particle_t particle;
	struct particle_list_element_s* prev;
	struct particle_list_element_s* next;
} particle_list_element_t;


typedef struct
{
	particle_list_element_t* particle_list;
	uint32_t particle_count;
} particle_grid_element_t;

typedef particle_grid_element_t panel_t[PARTICLE_GRID_X][PARTICLE_GRID_Y];
\end{lstlisting}

\begin{wrapfigure}{r}{0.5\textwidth}
	\centering
	\includegraphics[width=.45\linewidth]{IMG_20190409_212603.jpg}
	\caption{Würfelmodell mit Koordinatensystem der LED-Panels und der 2D-Arrays}
	\label{fig:sph:cube_model}
\end{wrapfigure}

Um den Würfel darzustellen werden nun sechs der 2D-Arrays genutzt. Wenn die neue Position eines Partikels nun die Außengrenzen des Arrays überschreiten würde, so wird es auf die andere Würfelseite übertragen. Die Implementierung der Simulation wurde während der Entwicklung auf einem Linux-PC getestet. Hierzu wurde sie in ein Programm eingebettet, die Position der einzelnen Partikel ausließt und über X11 auf dem Bildschirm darstellt. Da die Simulation nur mit elementaren Rechenoperationen arbeitet konnte sie im Anschluss problemlos in das ARM-Programm eingefügt werden. Dieses stellt nun die Anzahl der Partikel in einem Feld der 2D-Arrays als Farbschattierung auf dem LED-Display dar.

\section{Rechenbeschleunigung durch RaspberryPi 1B}

\clearpage
\section{Accelerometer}
%Beschreibung der Implementierung der Kommunikation mit & Auswertung der Werte von dem Gyro

\subsection{Hardware}
Für die Ermittlung der Position des Würfels wurde ein digital Beschleunigungsmesser verwendet. Für ein Projekt standen zwei Beschleunigungsmesser zur Verfügung: BNO055 von Adafruit und ADXL345 von Analog Devices. Zwar ein Beschleunigungsmesser von Adafruit mächtiger ist und mehr Messungen erlaubt, das Projekt wurde am Ende unter der Verwendung des Beschleunigungsmessers ADXL345 geschrieben. Nach mehreren Versuchen, mit Adafruit zu arbeiten, wurde es festgestellt, dass Beschleunigungsmesser nicht arbeitsfähig ist. In alle Registern des Bausteins anstatt der vorprogrammierte Daten liegt nur die Zahl "0xe5", was laut der Hersteller zeigt, dass Registern irgendwie geleert wurden.\\

Für die Verwendung des ADXL345 sollte man in Programm nur die Adressen ändern und richtig alle Pins belegen. Die ALT ADDRESS wurde auf 1 gesetzt und die 7-Bit-I2C-Adresse für das Gerät wurde somit als "0x1D" festgelegt. In Programm wurde direkt dir entsprechende 8-Bir-I2C-Adresse "0x3A" verwendet und die notwendige Änderung der Adresse in "0x3B" aus "0x3A" wurde gerade in I2C Routine erledigt. So kann man unterscheiden, ob es Lesen oder Schreiben Operation durchführen sollen werden. \\

Es gibt keine internen Pull-Up- oder Pull-Down-Widerstände für nicht verwendete Pins. Daher gibt es keinen bekannten Status oder Standardstatus für den CS- oder ALT ADDRESS-Pin, wenn er potentialfrei bleibt oder nicht verbunden ist. Es ist erforderlich, dass der CS-Pin an VDD  angeschlossen ist und dass der ALT ADDRESS-Pin bei Verwendung von I2C entweder an VDD oder GND angeschlossen ist (wir haben an VDD angeschlossen).\\

Die gemessene Daten werden aus die Register 0x32 bis Register 0x37 gelesen. Diese sechs Bytes (Register 0x32 bis Register 0x37) sind jeweils acht Bits und enthalten die Ausgangsdaten für jede Achse. Register 0x32 und Register 0x33 enthalten die Ausgangsdaten für die X-Achse, Register 0x34 und Register 0x35 die Ausgangsdaten für die Y-Achse und Register 0x36 und Register 0x37 die Ausgangsdaten für die Z-Achse. \\

Die Ausgangsdaten sind zwei Komplemente, mit DATAx0 als niedrigstwertigem Byte und DATAx1 als höchstwertigem Byte, wobei x X, Y oder Z darstellt. Um die Daten richtig zu lesen, wurde es ein Mehrbyte-Lesen von 2 Register durchgeführt und die Daten aus zwei Lesevorgängen nach dem Lesen als 16-Bit Ausgangsdaten dargestellt. 

\begin{lstlisting}
uint16_t i2c16bit = 0x00;
ReadLenght = 2;
GlobalI2CAddr = addr;
I2CMasterBuffer[0] = regs[0];
I2CMasterBuffer[1] = regs[1];
...
// merge the data from two registers
i2c16bit = i2c16bit | I2CReadBuffer[1]; // [REG0, REG1]: REG1 as MSB
i2c16bit = i2c16bit << 8;

i2c16bit = i2c16bit | I2CReadBuffer[0]; // [REG0, REG1]: REG0 as LSB

\end{lstlisting}

\subsection{I2C und Keil Board}
Für unseres Projekt haben wir I2C Block auf dem Keil Board in Master Mode programmiert. Mit Master-Sendermodus werden Daten vom Master (Keil Board) zum Slave (Beschleunigungsmesser) übertragen. So erlaubt uns den Beschleunigungsmesser zu initialisieren. \\

Im Master-Empfängermodus werden Daten von einem Slave-Sender empfangen. Nach der Initialisierung des Beschleunigungsmessers wird es immer weiter in diesem Modus gearbeitet, da wir nur ständig die ermittelte Position ablesen wollen. \\

Bevor der Mastersender-Modus aufgerufen werden kann, muss das I2CONSET-Register initialisiert werden

\subsection{Software}

\begin{lstlisting}
unsigned int ADXLI2CAdresss = 0x3A;

\end{lstlisting}


\subsection{Probleme und Lösungen}
\section{Würfel mit LED-Panels}
%Beschreibung der Implementierung der Darstellungen auf den Displays
Die Komponente Würfel empfängt Daten der Panelinformationen von der Komponente Flüssigkeitssimulation und beleuchtet die Panels entsprechend dieser Daten. Die Anzahl der Teilchen pro Pixel dient als Intensität der blauen Farbe.

\subsection{Hardware}
Der RGB-Würfel wird aus sechs 32$\times$32 RGB Matrix LED-Panels aufgebaut. Ein Panel benötigt den Strom von $2A$. Die Panels werden anhand einer Stromversorgung geschaltet. Ein Panel wird über 16 Kabel direkt mit dem Board verbunden und die restlichen fünf Panels werden nebeneinander anhand Flachbandkabel verbunden (sogenannte Daisy-Chain). Abbildung \ref{fig:assembly:dice} stellt den gesamten Aufbau der Komponente Würfel dar.
\begin{figure}
	\centering
	\includegraphics[width=0.95\linewidth]{example-image}
	\caption{Aufbau von sechs 32$\times$32 RGB Matrix LED-Panels}
	\label{fig:assembly:dice}
\end{figure}

\subsection{Mechanismus und Konfiguration einer Anzeige}
Eine Anzeige hat 32 LEDs in einer Reihe beziehungsweise $32 \times 32 = 1024$ LEDs. Diese werden in 16 Teile abgeschnitten. Erster Teil ist für die 1.- und 16. Zeile, zweiter Teil ist für die 2.- und 17. Zeile bis hin zum sechzehnten Teil für die 16.- und 32. Zeile.

Die Information eines Pixels wird per Takt mittels des Schiebregisters horizontal verschoben. Werden mehrere Panels verkettet, wird die Information zum weiteren Panel verschoben. Das heißt, die verketteten Panels können als ein Panel mit den breiteren Spalten betrachtet werden. \\

\emph{<Schritte>}
\begin{enumerate}
	\item Setzen alle Pins auf \texttt{Low} (anfang)
	\item \texttt{R1,G1,B1,R2,G2,B2} Pins auf \texttt{High/Low} setzen
	\item \texttt{CLK} Pin auf \texttt{High} setzen
	\item \texttt{CLK} Pin auf \texttt{Low} setzen
	\item Schritte 2-4 für alle Spalten wiederholen
	\item Mit Adresspins \texttt{ABCD} eine gezielte Zeile auswählen
	\item \texttt{OE} Pin auf \texttt{High} setzen
	\item \texttt{LAT} Pin auf \texttt{High} setzen (erleuchtet LED)
	\item \texttt{LAT} Pin auf \texttt{Low} setzen
	\item \texttt{OE} Pin auf \texttt{Low} setzen
\end{enumerate}

Jeder Zyklus leuchtet LEDs und schaltet diese gleich aus. Die Unterfunktion \texttt{refresh} muss daher kontinuierlich aufgerufen werden.

\subsection{Software}

\chapter{Ergebnis}


\clearpage




%%%%%%%%%%%%%%%%%%%%%%%%%%%%%%%%%%%%%%%%%%%%%%%%%%%%%%%%%%%%%%%
%% Literaturverzeichnis

\end{document}
